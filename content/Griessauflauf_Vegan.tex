% !TEX root = ../CookBook.tex

% ====== Rezeptname und die Quelle ======
\begin{recipe}[]{Griessauflauf Vegan}{}{}

% ====== Zeit, Personen und Schaerfe ======
\timerecipe{2}              % Zubereitungszeit in Stunden
\personcount{5}             % Personenanzahl
\spice{0}                   % Schaerfe von 0 bis 5

% ====== Zutaten ======
\ingredient{2.2 \l Sojamilch}
\ingredient{1 \TL Salz}
\ingredient{60 \g Margarine}
\ingredient{350 \g Griess}
\ingredient{80 \g Margarine}
\ingredient{300 \ml Agavendicksaft}
\ingredient{1-2 Zitronenschale(n)}
\ingredient{1 \TL Backpulver}
\ingredient{120 \g geriebene Mandeln}
\ingredient{120 \g Rosinen}
\ingredient{500 \g nature Sojajogurt}

% ====== Zubereitung ======    
\step
Sojamilch mit Salz und 60 \g Margarine aufk\"ocheln.

\step
Griess einr\"uhren und etwa 15 Minuten kochen lassen.

\step
Die restliche Margarine, den Agavendicksaft und die Zitronenschale darunter r\"uhren und danach abk\"uhlen lassen.

\step
Das Backpulver, die Mandeln und die Rosinen in die Masse einmischen.

\step
Das Sojajogurt dazumischen. Gut vermischen, damit die Masse gleichm\"assig ist.

\step
1-2 h backen bei 160 - 180 Grad C.

% ====== Bild ======
%\graphic{pictures/Bild.jpg}
\end{recipe}
