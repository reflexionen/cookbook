% !TEX root = ../CookBook.tex

% ====== Rezeptname und die Quelle ======
\begin{recipe}[]{Hummus}{}{}

% ====== Zeit, Personen und Schaerfe ======

% ====== Zutaten ======
\ingredient{300 \g getrocknete Kichererbsen}
\ingredient{1 Zitrone}
\ingredient{4 \EL Oliven\"ol}
\ingredient{1/2 \TL gemahlener Kreuzk\"ummel}
\ingredient{Salz}
\ingredient{2 \EL weisses Tahin}

% ====== Zubereitung ======
\step
Die Kichererbsen \"uber Nacht in kaltem Wasser einweichen.

\step
Am n\"achsten Tag die Kichererbsen abgiessen. Danach 1 h in Wasser kochen, bis sie weich sind.

\tippboxtip{Die Kichererbsen nach dem Kochen noch sch\"alen.}

\step
Mit dem P\"urierstab fein p\"urieren. Je feiner, desto besser wird der Hummus.

\step
Den Brei mit 100 \ml Wasser verd\"unnen. Mit Zitronensaft, Oliven\"ol, Kreuzk\"ummel und Salz w\"urzen.

\step
Das Tahin dazu mischen. Es bindet die ganze Masse zusammen.

\tippboxtip{
Wenns schneller gehen soll, kann auch Kichererbsenmehl verwendet werden. Dieses einfach mit der dreifachen Menge Wasser kochen und danach einen Moment stehen lassen, aber es sollte noch warm sein. \\
Danach die Zutaten einarbeiten (Kreuzk\"ummel, Oliven\"ol, Salz, ...).
}

% ====== Bild ======
% Grafik fuer das Rezept koennen so eingefuegt werden:
% wenn kein Bild vorhanden ist, bitte diese Zeile auskommentiert lassen.
%\graphic{pictures/Bild.jpg}
\end{recipe}