% !TEX root = ../CookBook.tex

% ====== Rezeptname und die Quelle ======
\begin{recipe}[]{Spagetti Carbonara mit Fleisch}{}{}

% ====== Zeit, Personen und Schaerfe ======
\timerecipe{1}              % Zubereitungszeit in Stunden
\personcount{5}             % Personenanzahl

% ====== Zutaten ======
\ingredient{1 \kg~Spagetti}
\ingredient{2 Zwiebeln}
\ingredient{5 Zehen Knoblauch}
\ingredient{400 \g Speckw\"urfeli}
\ingredient{5 \dl~Halbrahm}
\ingredient{5 Eier}
\ingredient{500 \g Reibk\"ase}
\ingredient{2 \EL~Salz}
\ingredient{wenig Streuw\"urze und Muskat}
\ingredient{viel Pfeffer}

% ====== Zubereitung ======
\step%
Das Wasser f\"ur die Spagetti aufkochen mit 2 \EL~Salz. Wenn das Wasser kocht,
die Spagetti dazu geben.

\step%
Die Zwiebeln und den Knoblauch klein schneiden und dann anbraten. Wenn die
Zwiebeln leicht golden werden, die Speckw\"urfeli dazugeben. Mit Salz und viel
Pfeffer w\"urzen.

\step%
Sobald die Speckw\"urfeli gut angebraten sind, die Platte ausschalten und die
Eier dazu schlagen. Kurz warten, bis nicht mehr alle Eiweiss fl\"ussig ist,
sondern etwa die h\"alfte schon leicht fest gekocht ist, dann den Rahm dazu
geben.

\step%
Den Reibk\"ase in die Sosse geben und das ganze verr\"uhren. Mit Streuw\"urze,
Muskat und Pfeffer abschmecken und wenn die Spagetti fertig sind, dazu geben.

\tippbox{%
Dazu Reibk\"ase servieren
}

% ====== Bild ======
%\graphic{pictures/Bild.jpg}
\end{recipe}
