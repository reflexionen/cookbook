% !TEX root = ../CookBook.tex

% ====== Rezeptname und die Quelle ======
\begin{recipe}[]{Rotes Kokos-Curry}{}{}

% ====== Zeit, Personen und Schaerfe ======
\timerecipe{2.5}            % Zubereitungszeit in Stunden
%\timerecipe[Minuten]{10}   % oder in Minuten
\personcount{6}             % Personenanzahl
\spice{2}                   % Schaerfe von 0 bis 5

% ====== Zutaten ======
\ingredient{200 \g trockene Sojakl\"osse}
\ingredient{400 \ml Gem\"use Bouillon}
\ingredient{2 \EL Mehl}
\ingredient{3 rote Randen}
\ingredient{2-3 Chilischoten}
\ingredient{Kreuzk\"ummel}
\ingredient{1 \TL Kurkuma}
\ingredient{2 \EL Tamarindensaft}
\ingredient{800 \ml Kokosmilch}
\ingredient{3 \EL Pflanzen\"ol}
\ingredient{1 \EL Senfk\"orner}
\ingredient{4 rote Zwiebeln}
\ingredient{500 \g Champignons}
\ingredient{250 \g gr\"une Bohnen}
\ingredient{3-4 Kaffirlimettenbl\"atter}
\ingredient{Salz, Pfeffer}

% ====== Zubereitung ======
\step
Die Sojakl\"osse in der heissen Bouillon 15 Minuten einweichen, abgiessen und mit Mehl best\"auben.

\step
Die Randen in Wasser etwa 20-30 Minuten kochen, bis sie etwas weich sind, danach sch\"alen und in 1 cm grosse W\"urfel schneiden.

\step
Chilischoten, Kreuzk\"ummel, Tamarindensaft und Kurkuma in einer Schale zusammenmixen und dr\"ucken, bis eine Paste daraus wird. Dies funktioniert meistens in einem M\"orser besser.

\step
Die Zwiebeln in einer Pfanne kurz anbraten und dann die Paste mit den Randen dazu geben, die Kokosmilch dazu schmeissen und das ganze aufkochen und 5 Minuten k\"ocheln lassen. Danach die Champignons, die Bohnen und die Kaffirlimettenbl\"atter auch noch in die Pfanne f\"ullen.

\step
Die Senfk\"orner in heissem \"Ol 10 Sekunden anbraten. Die Sojakl\"osse dazu geben und warten bis sie sich in der Sauna f\"uhlen und von selber braun werden. Mit Salz und Pfeffer w\"urzen.

\step
Die beiden Pfannen zusammen mischen und weitere 5-10 Minuten k\"ocheln lassen, bis es die gew\"unschte Konsistenz erreicht hat.

\tippboxtip{
Dazu Reis oder Kartoffeln servieren.
}

% ====== Bild ======
% Grafik fuer das Rezept koennen so eingefuegt werden:
% wenn kein Bild vorhanden ist, bitte diese Zeile auskommentiert lassen.
%\graphic{pictures/Bild.jpg}
\end{recipe}