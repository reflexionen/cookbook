% !TEX root = ../CookBook.tex

% ====== Rezeptname und die Quelle ======
\begin{recipe}[]{Griessauflauf}{}{}

% ====== Zeit, Personen und Schaerfe ======
\timerecipe{2}              % Zubereitungszeit in Stunden
\personcount{5}             % Personenanzahl

% ====== Zutaten ======
\ingredient{2.2 \l Milch}
\ingredient{1 \TL Salz}
\ingredient{60 \g Margarine}
\ingredient{350 \g Griess}
\ingredient{80 \g Margarine}
\ingredient{160 \g Zucker}
\ingredient{1-2 Zitronenschale(n)}
\ingredient{1 \TL Backpulver}
\ingredient{120 \g geriebene Mandeln}
\ingredient{120 \g Rosinen}
\ingredient{8 Eier}

% ====== Zubereitung ======
\step
Sojamilch mit Salz und 60 \g Margarine aufk\"ocheln.

\step
Griess einr\"uhren und etwa 15 Minuten kochen lassen.

\step
Den Griess abk\"uhlen lassen.

\step
Die restliche Margarine, den Zucker, das Eigelb und die Zitronenschale darunter r\"uhren.

\step
Das Backpulver, die Mandeln, die Rosinen und das Eiweiss in kleinen Schritten unter die Masse ziehen.

\step
1 h backen bei 180 - 200 \textdegree C.

% ====== Bild ======
%\graphic{pictures/Bild.jpg}
\end{recipe}
