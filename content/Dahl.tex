% !TEX root = ../CookBook.tex

% ====== Rezeptname und die Quelle ======
\begin{recipe}[]{Dahl}{Vegan lecker lecker}{}

% ====== Zeit, Personen und Schaerfe ======
\timerecipe{1}              % Zubereitungszeit in Stunden
\personcount{2}             % Personenanzahl
\spice{3}                   % Schaerfe von 0 bis 5

% ====== Zutaten ======
\ingredient{25 \ml \"Ol}
\ingredient{2 Knoblauchzehen}
\ingredient{2 Zwiebeln}
\ingredient{1 \TL Kurkuma}
\ingredient{2 \TL Garam Masala}
\ingredient{3 scharfe Chilischoten}
\ingredient{1 \TL gemahlener Kreuzk\"ummel}
\ingredient{800 \g Pelatti}
\ingredient{250 \g rote Linsen}
\ingredient{2 \TL Zitronensaft}
\ingredient{600 \ml Gem\"usebr\"uhe}
\ingredient{400 \ml Kokosmilch}
\ingredient{Salz und Pfeffer}

% ====== Zubereitung ======    
\step
Das \"Ol in einem grossen Topf erhitzen. Knoblauch und Zwiebeln klein schneiden und dazugeben und golbbraun anbraten. Kurkuma, Garam Masala und Kreuzk\"ummel hinzuf\"ugen und weitere 30 Sekunden anbraten.

\step
Die Tomaten w\"urfeln und mit roten Linsen, Gem\"usebr\"uhe und Chilischoten hinzugeben und alles zum Kochen bringen.

\tippboxtip{
Wers weniger scharf will, die Chilischoten durch etwas Cayennepfeffer ersetzen.
}

\step
Auf kleiner Hitze die Suppe nicht abgedeckt 25-30 Minuten k\"ocheln lassen, bis die Linsen weich gekocht sind.
Gelegentlich r\"uhren, damit die Linsen nicht anhocken.

\step
Mit Salz und Pfeffer abschmecken, etwas Zitronensaft und die Kokosmilch dazu geben. Weitere 5 Minuten k\"ocheln lassen.

\tippboxtip{
Dazu passt sehr gut Reis.
}

% ====== Bild ======
%\graphic{pictures/Bild.jpg}
\end{recipe}
