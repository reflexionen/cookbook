% !TEX root = ../CookBook.tex

% ====== Rezeptname und die Quelle ======
\begin{recipe}[]{Dahl}{Vegan lecker lecker}{}

% ====== Zeit, Personen und Schaerfe ======
\timerecipe{1}              % Zubereitungszeit in Stunden
\personcount{5}             % Personenanzahl
\spice{4}                   % Schaerfe von 0 bis 5

% ====== Zutaten ======
\ingredient{25 \ml \"Ol}
\ingredient{6 Knoblauchzehen}
\ingredient{3 Zwiebeln}
\ingredient{2 \TL Kurkuma}
\ingredient{4 \TL Garam Masala}
\ingredient{10 scharfe Chilischoten}
\ingredient{2 \TL gemahlener Kreuzk\"ummel}
\ingredient{1.6 \kg Pelatti}
\ingredient{500 \g rote Linsen}
\ingredient{4 \TL Zitronensaft}
\ingredient{600 \ml Gem\"usebr\"uhe}
\ingredient{800 \ml Kokosmilch}
\ingredient{Pfeffer}
\ingredient{Cayennepfeffer}

% ====== Zubereitung ======    
\step
Das \"Ol in einem grossen Topf erhitzen. Knoblauch und Zwiebeln klein schneiden und dazugeben und golbbraun anbraten.

\step
Kurkuma, Garam Masala und Kreuzk\"ummel hinzuf\"ugen und weitere 30 Sekunden anbraten.

\step
Rote Linsen hinzuf\"ugen und kurz and\"unsten.

\step
Die Tomaten w\"urfeln, Gem\"usebr\"uhe und Chilischoten hinzugeben und alles zum Kochen bringen.

\tippboxtip{
R\"uebli kommen auch sehr gut darin. Einfach mit den Tomaten hinzuf\"ugen, damit sie noch weich kochen k\"onnen.
}

\step
Auf kleiner Hitze das Ganze nicht abgedeckt 25-30 Minuten k\"ocheln lassen, bis die Linsen weich gekocht sind.
Gelegentlich r\"uhren, damit die Linsen nicht anhocken.

\step
Mit Pfeffer und Cayennepfeffer abschmecken und die Kokosmilch dazu geben. Weitere 5 Minuten k\"ocheln lassen.

\step
Vor dem servieren noch etwas Zitronensaft dazu.

\tippboxtip{
Reis ist die perfekte Erg\"anzung, vorausgesetzt das Brot ist ausgegangen. Mit Fabio brauchts immer viel Reis.
}

% ====== Bild ======
%\graphic{pictures/Bild.jpg}
\end{recipe}
