% !TEX root = ../CookBook.tex

% ====== Rezeptname und die Quelle ======
\begin{recipe}[]{Veganisierte Spagetti Carbonara}{}{}

% ====== Zeit, Personen und Schaerfe ======
\timerecipe{1}              % Zubereitungszeit in Stunden
\personcount{5}             % Personenanzahl
\spice{0}                   % Schaerfe von 0 bis 5

% ====== Zutaten ======
\ingredient{70 \g gemahlene Paran\"usse}
\ingredient{3x R\"auchertofu}
\ingredient{5 \dl~Sojamilch}
\ingredient{300 \ml~Sojarahm}
\ingredient{etwas Paprika}
\ingredient{etwas Cayennepfeffer}
\ingredient{3 \TL~Senf}
\ingredient{10 geh\"aufte \EL~Edelhefeflocken}
\ingredient{etwas Muskatnuss}
\ingredient{1/2 \TL~Bouillon}
\ingredient{etwas Sojasauce}
\ingredient{2--3 gr. Zwiebeln}
\ingredient{etwas Zitronensaft}
\ingredient{60 \g Margarine}
\ingredient{1 \kg~Spagetti}

% ====== Zubereitung ======
\step%
Paran\"usse malen bis sie sich in feine St\"uckchen verwandeln.

\step%
Den R\"auchertofu in kleine St\"ucke schneiden und die Zwiebel mit dem Messer
bedr\"angen.

\tippbox{Der Tofu hat sehr lange.}

\step%
Den Tofu anbraten bis er braun ist. Mit etwas Pfeffer w\"urzen und mit
Sojasauce abl\"oschen.

\step%
Die Zwiebeln und die Paran\"usse in Margarine in einer Pfanne anschwitzen.
Etwas Cayennepfeffer, Paprika und die Bouillon hinzuf\"ugen. Nach kurzer Zeit
die Soyamilch und den Sojarahm hinzuf\"ugen und mit dem Senf verkuppeln.

\step%
Nach wenigen Minuten die Edelhefeflocken dazu geben und das ganze eindicken lassen.
Mit Pfeffer und Muskatnuss abschmecken.

\step%
Die Sauce mit dem Tofu vermischen und noch den Zitronensaft dazugeben.

\tippboxtip{Die Bouillon mit R\"auchersalz ersetzen.}
\tippboxtip{Die Margerine durch Raps\"ol mit Buttergeschmack ersetzen.}

% ====== Bild ======
%\graphic{pictures/Bild.jpg}
\end{recipe}
