% !TEX root = ../CookBook.tex

% ====== Rezeptname und die Quelle ======
\begin{recipe}[]{Linseneintopf}{}{}

% ====== Zeit, Personen und Schaerfe ======
\timerecipe{1}              % Zubereitungszeit in Stunden
%\timerecipe[Minuten]{10}   % oder in Minuten
\personcount{4}             % Personenanzahl

% ====== Zutaten ======
\ingredient{250 \g Linsen}
\ingredient{250 \g eingelegte Kichererbsen}
\ingredient{200 \g weisse Bohnen}
\ingredient{2 grosse Karotten}
\ingredient{1 Sellerie oder Peterliwurz}
\ingredient{2 grosse Tomaten}
\ingredient{2 grosse Zwiebel}
\ingredient{2-3 Zehen Knoblauch}
\ingredient{1 \l Gem\"usebr\"uhe}
\ingredient{1 \dl Weisswein}
\ingredient{Liebst\"ockel}
\ingredient{Petersilie}
\ingredient{Salz und Pfeffer}
\ingredient{Zitronensaft}

% ====== Zubereitung ======
\step
Die Linsen und die Bohnen zuerst waschen und dann 30 Minuten in warmem Wasser einweichen lassen.

\step
Das Gem\"use in kleine St\"ucke schneiden. Die Karotten kommen gut, wenn sie zuerst l\"angs gevierteilt werden und dann noch in St\"ucke von ungef\"ahr einem halben Zentimeter geschnitten werden. Den Knoblauch in feine Scheiben schneiden.

\step
Einen Topf erhitzen und den Knoblauch und die Zwiebeln in \"Ol leicht anbraten. Anschliessend die Karotten und den Sellerie hinzugeben und eine Weile d\"unsten lassen. Mit wenig Salz w\"urzen.

\step
Den Weisswein zum Abl\"oschen verwenden und dann aufk\"ocheln lassen. Dann die Linsen und Bohnen hinzugeben.

\step
Die Tomaten in kleine St\"ucke verhauen und in den Topf geben. Die Br\"uhe dann ungef\"ahr 20 Minuten k\"ocheln lassen.

\step
Die Kichererbsen hinzugeben und nochmals weitere 20 Minuten k\"ocheln lassen.

\step
Die Suppe mit Pfeffer und Salz abschmecken. Vor dem Servieren die gehackte Petersilie, den Liebst\"ockel und den Zitronensaft hinzugeben.

\tippbox{
Die Linsensuppe kann gut am Vortag vorbereitet werden, da sie nach dem Aufw\"armen am n\"achsten Tag noch besser schmeckt. 
}

% ====== Bild ======
%\graphic{pictures/Bild.jpg}
\end{recipe}