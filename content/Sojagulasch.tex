% !TEX root = ../CookBook.tex

% ====== Rezeptname und die Quelle ======
\begin{recipe}[]{Sojagulasch}{}{}

% ====== Zeit, Personen und Schaerfe ======
%\timerecipe{1}             % Zubereitungszeit in Stunden
%\timerecipe[Minuten]{10}   % oder in Minuten
%\personcount{4}            % Personenanzahl
\spice{2}                   % Schaerfe von 0 bis 5

% ====== Zutaten ======
\ingredient{Soyana Sojakl\"osse}
\ingredient{Gem\"usebr\"uhe}
\ingredient{Mehl}
\ingredient{gr\"une Linsen}
\ingredient{5-6 Karotten}
\ingredient{800 \g Pelatti}
\ingredient{Chilis}
\ingredient{Cayennepfeffer}
\ingredient{Salz}
\ingredient{Kartoffeln}

% ====== Zubereitung ======
\step
Die Kartoffeln in gleich grosse St\"uckchen schneiden und in Wasser kochen.

\step
Die Sojakl\"osse 15 min in Gem\"usebr\"uhe einlegen.

\step
Die Kl\"osse abtropfen lassen und danach mit Mehl best\"auben.

\step
Die Sojakl\"osse anbraten und mit Cayennepfeffer, Salz und Paprika w\"urzen.

\step
Karotten und Chilis klein schneiden.

\step
Linsen und die geschnittenen Karotten, Pelatti und die Chilis dazu geben und dann l\"angere Zeit kochen lassen, bis die Karotten weich gekocht sind.

\step
Die weich gekochten Kartoffeln noch dazumischen und servieren.

% ====== Bild ======
% Grafik fuer das Rezept koennen so eingefuegt werden:
% wenn kein Bild vorhanden ist, bitte diese Zeile auskommentiert lassen.
%\graphic{pictures/Bild.jpg}
\end{recipe}