% !TEX root = ../CookBook.tex

% ====== Rezeptname und die Quelle ======
\begin{recipe}[]{Tofu stroganov}{Vegan lecker lecker}{}

% ====== Zeit, Personen und Schaerfe ======
\timerecipe{1.5}              % Zubereitungszeit in Stunden
\personcount{6}             % Personenanzahl
\spice{0}                   % Schaerfe von 0 bis 5

% ====== Zutaten ======
\ingredient{600 \g Tofu}
\ingredient{6 \EL Sojasauce}
\ingredient{4 \TL Sambal Oelek}
\ingredient{2 \EL Oliven\"ol}
\ingredient{2 gr. Zwiebel}
\ingredient{500 \g Champignons}
\ingredient{6 \EL Tomatenmark}
\ingredient{2 \TL Scharfer Senf}
\ingredient{400 \ml Sojarahm}
\ingredient{Rotwein, Cognac}
\ingredient{Rosmarin}
\ingredient{Pfeffer, Cayennepfeffer}
\ingredient{Paprika}
\ingredient{Zitronensaft}

% ====== Zubereitung ======    
\step
Den Tofu w\"urfeln. Die Sojasauce, Sambal Oelek, Oliven\"ol, zwei \TL scharfer Senf und etwas Pfeffer, Cayennepfeffer und Paprika in einer mittelgrossen Sch\"ussel zu einer Marinade vermischen. Dazu den Tofu geben und f\"ur 30 Minuten ziehen lassen. Hin und wieder umr\"uhren.

\step
Die Zwiebeln und Champignons klein schneiden und in einer Pfanne anbraten. Mit Rotwein und Cognac abl\"oschen, danach Rosmarin hinzuf\"ugen.

\step
Dann den Tofu separat in heissem \"Ol \ca 15 Minuten anbraten.

\step
5 Minuten vor Schluss Sojarahm und Tomatenmark zu den Zwiebeln und Pilze hinzuf\"ugen. \\
Den Tofu zum ganzen dazu geben und weitere 5 Minuten k\"ocheln lassen.

\tippboxtip{
Den Tofu mit Seitangeschnetzeltem ersetzen.
}

\tippboxtip{
Mit Kartoffeln oder Teigwaren servieren.
}

% ====== Bild ======
%\graphic{pictures/Bild.jpg}
\end{recipe}
