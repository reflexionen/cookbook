% !TEX root = ../CookBook.tex

% ====== Rezeptname und die Quelle ======
\begin{recipe}[]{Tofu stroganov}{Vegan lecker lecker}{}

% ====== Zeit, Personen und Schaerfe ======
\timerecipe{1}              % Zubereitungszeit in Stunden
\personcount{4}             % Personenanzahl
\spice{0}                   % Schaerfe von 0 bis 5

% ====== Zutaten ======
\ingredient{250 \g Tofu}
\ingredient{3 \EL Sojasauce}
\ingredient{1 \TL Sambal Oelek}
\ingredient{1 \EL Oliven\"ol}
\ingredient{1 gr. Zwiebel}
\ingredient{200 \g Champignons}
\ingredient{3 \EL Tomatenmark}
\ingredient{200 \ml Sojarahm}
\ingredient{Rotwein}
\ingredient{Rosmarin}
\ingredient{Cognac}

% ====== Zubereitung ======    
\step
Den Tofu w\"urfeln. Die Sojasauce, Sambal Oelek und Oliven\"ol in einer mittelgrossen Sch\"ussel zu einer Marinade vermischen. Dazu den Tofu geben und f\"ur 30 Minuten ziehen lassen. Hin und wieder umr\"uhren.

\step
Die Zwiebeln und Champignons klein schneiden und in einer Pfanne anbraten.

\step
Dann den Tofu separat in heissem \"Ol \ca 5 Minuten anbraten. Die Zwiebeln und Pilze hinzuf\"ugen. Mit Rotwein abl\"oschen. Dann das Tomatenmark und den Rosmarin hinzuf\"ugen und unterr\"uhren. Kurz aufkochen, dann die Hitze reduzieren. Den Sojarahm hinzugeben und weitere 5 Minuten k\"ocheln lassen.

\tippboxtip{
Mit Kartoffeln, Reis oder Teigwaren servieren.
}

% ====== Bild ======
%\graphic{pictures/Bild.jpg}
\end{recipe}
