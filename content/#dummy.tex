% !TEX root = ../CookBook.tex

% ====== Rezeptname und die Quelle ======
\begin{recipe}[]{REZEPTNAME}{AUTHOR}{FOTOGRAPH}

% ====== Zeit, Personen und Schaerfe ======
%\timerecipe{1}             % Zubereitungszeit in Stunden
%\timerecipe[Minuten]{10}   % oder in Minuten
%\personcount{4}            % Personenanzahl
%\spice{3}                  % Schaerfe von 0 bis 5

% ====== Zutaten ======
%\ingredientpart{Zutaten fuer einen Teil des Ganzen}
\ingredient{2 frische Chilis}
\ingredient{2 EL Ketchup}

% ====== Zubereitung ======
\step
hier den zubereitungsschritt beschreiben

\step
hier den zubereitungsschritt beschreiben

\tippbox{
Dies ist ein Hinweis um noch ein paar Zusatzinformationen und weitere Anregungen dem Rezept beizuf\"ugen.
}

\step
hier den zubereitungsschritt schreiben

% Tipp in extra Rahmen mit dem Wort "Tipp:" am Anfang
\tippboxtip{
Dies ist ein umrahmter Tipp um noch ein paar Zusatzinformationen und weitere Anregungen dem Rezept beizuf\"ugen.
}

% ====== Bild ======
% Grafik fuer das Rezept koennen so eingefuegt werden:
% wenn kein Bild vorhanden ist, bitte diese Zeile auskommentiert lassen.
%\graphic{pictures/Bild.jpg}
\end{recipe}
