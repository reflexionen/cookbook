% !TEX root = ../CookBook.tex

% ====== Rezeptname und die Quelle ======
\begin{recipe}[]{Soyana Curry}{Soyana}{}

% ====== Zeit, Personen und Schaerfe ======
\timerecipe{2}              % Zubereitungszeit in Stunden
\personcount{4}             % Personenanzahl
\spice{2}                   % Schaerfe von 0 bis 5

% ====== Zutaten ======
\ingredient{400 \g Sojakl\"osse von Soyana}
\ingredient{1.8 \l Bouillon}
\ingredient{2 gr. Zwiebeln}
\ingredient{4 Knoblauchzehen}
\ingredient{0.5 \l Sojarahm}
\ingredient{100 \g Zibeben (Wyberi)}
\ingredient{100 \g Pinienkerne}
\ingredient{2 \"Apfel}
\ingredient{1 Orange}
\ingredient{2 \TL~Senf}
\ingredient{6 \EL~Mehl}
\ingredient{Weisswein}
\ingredient{viel Curry Pulver}
\ingredient{1 \EL~Kreuzk\"ummel}
\ingredient{1 \EL~Garamasala}
\ingredient{1 \EL~Paprika}
\ingredient{1 \EL~Cayenne Pfeffer}

\tippbox{Die Zibeben k\"onnen mit Sultaninen ersetzten werden.}

% ====== Zubereitung ======
\step%
Die Sojakl\"osse in der noch warmen Bouillon (darf sehr starke Bouillon sein)
15 Minuten aufgehen lassen. Danach ausdr\"ucken und dann mit Mehl best\"auben
(je mehr Mehl desto mastiger das Essen).

\step%
Die \"Apfel und Orangen klein Schneiden, bis sie die gew\"unschte Gr\"osse
erreichen.

\step%
Die Zwiebeln zu kleinen W\"urfeln verarbeiten.

\step%
Die Sojakl\"osse anbraten bis sie braun sind, danach die Zwiebeln und den
Knoblauch mit den Kl\"ossen verheiraten.

\step%
Gew\"urze, Fr\"uchte und Zibeben beigeben und alle leicht aufk\"ocheln lassen.
Danach mit Weisswein abl\"oschen.

\step%
Sojarahm beigeben und das ganze mit noch mehr Gew\"urze abschmecken.

\tippboxtip{Dazu Reis servieren}
\tippboxtip{Die Sojakl\"osse nur knapp mit Bouillon bedecken und zus\"atzlich noch Sojasauce und Tomatenmark hinzugeben (gut umr/"uhren)}

% ====== Bild ======
% Grafik fuer das Rezept koennen so eingefuegt werden:
% wenn kein Bild vorhanden ist, bitte diese Zeile auskommentiert lassen.
%\graphic{pictures/Bild.jpg}
\end{recipe}
