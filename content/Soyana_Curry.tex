% !TEX root = ../CookBook.tex

% ====== Rezeptname und die Quelle ======
\begin{recipe}[]{Soyana Curry}{Soyana}{}

% ====== Zeit, Personen und Schaerfe ======
\timerecipe{1}              % Zubereitungszeit in Stunden
\personcount{4}             % Personenanzahl
\spice{2}                   % Schaerfe von 0 bis 5

% ====== Zutaten ======
\ingredient{400 \g Sojakl\"osse von Soyana}
\ingredient{1.8 \l Bouillon}
\ingredient{Mehl}
\ingredient{2 gr. Zwiebeln}
\ingredient{Zibeben (Wyberi)}
\ingredient{Pinienkerne}
\ingredient{0.5 \l Sojarahm}
\ingredient{Weisswein}
\ingredient{Curry Pulver}
\ingredient{Kreuzk\"ummel}
\ingredient{Garamasala}

\tippbox{Die Zibeben mit Sultaninen ersetzen.}

% ====== Zubereitung ======
\step
Die Sojakl\"osse in der noch warmen Buillon 15 Minuten aufgehen lassen. Danach ausdr\"ucken und dann mit Mehl best\"auben (je mehr Mehl desto mastiger das Essen).

\step
Die Sojakl\"osse anbraten.

\tippboxtip{Noch 2 \"Apfel und eine Orange in kleinen St\"uckchen dazu geben.}

\tippboxtip{Dazu Reis servieren}

% ====== Bild ======
% Grafik fuer das Rezept koennen so eingefuegt werden:
% wenn kein Bild vorhanden ist, bitte diese Zeile auskommentiert lassen.
%\graphic{pictures/Bild.jpg}
\end{recipe}
